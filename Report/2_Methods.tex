\newpage
\chapter{Methodology}
\section*{Search Criteria}
The data in meta-analyses originate from existing studies.
For this meta-analysis, researchers first compiled a body of relevant, existing literature from four databases (psycINFO, Medline, EMBASE, and Web of Science Core Collection; conducted March 13, 2023) by considering research on either social phobia (SAD) and meta-cognition or social phobia and HRV.
After duplicates were removed from the pool of papers, two raters screened titles and abstracts for inclusion and exclusion criteria.
Inclusion criteria consist of: studies with a socially anxious patient population, papers that specifically mention one of the instruments used to measure meta-cognition or HRV, and studies with patient populations under the age of 24.
Papers that include co-morbid disorders or where social anxiety is not considered separately in the results section were excluded.
After the remaining papers were screened by a second rater, 34 papers were retained for meta-cognition, and 54 papers were retained for HRV.
Of these, two additional raters independently screened 30 (meta-cognition) and 50 (HRV) papers for inclusion, with raters largely in agreement on paper inclusion or exclusion (Kappa inter-reliability: Meta-cognition = 0.902; HRV = 0.913). \\
For this initial meta-analysis, we restrict the scope of the study by considering only the literature on meta-cognition. These papers were then carefully read and relevant data extracted.

\section*{Data Information}
The data set compiled for the meta-analysis contains $26$ variables that characterize each of the $54$ papers included in this study. The variables included in the data set fall into $5$ different categories: \textbf{Paper information}, \textbf{Participant information}, \textbf{Social anxiety data}, and \textbf{Metacognition methodology}.
The \textbf{Paper information} variables contain the information necessary for identifying the paper, such as the authors, year of publication, and the study's country of origin. 
The \textbf{Participant information} category is composed of variables that describe the sample included in each study, such as sample size, gender and ethnic composition, and participant age.
The \textbf{Social anxiety data} variables capture data on the type of study conducted (correlational or group comparison), the type of instrument used to measure social anxiety (e.g. self-reported, clinician assessed) and recruitment measure (e.g. referred by therapists).
The \textbf{Meta-cognition methodology} category includes variables related to the type of Meta-cognition measure used in a study (trait or state) and information on the meta-cognition measurement instrument used (the number of questions included in it).
Finally, the \textbf{Results} variables include the results obtained from each study, giving information about the statistical analysis performed, the effect size, the type of measure used (effect size: d-measures or correlation: r-measures), and the significance of the results.
The $95\%$ confidence interval and the variance of the effect size estimates are also included.

\section*{Statistical Analysis}
A random-effects meta-analysis was performed to estimate the overall effect size of the relationship between meta-cognition and social anxiety.
Initially, the data set was cleaned by removing papers with missing information on the effect size.
Moreover, it was necessary to uniform the different effect size estimates type into a single one, since both means differences with standard deviations ($d$ effect size) and correlations ($r$ effect size) were included in the data set.
Therefore, the correlation coefficients were transformed into the corresponding effect size estimate and standard deviations using the following formulas reported by \cite{mathur_simple_2020}:
$$d = \frac{2r}{\sqrt{1 - r^2}}$$
$$\hat{SE}(d) = \frac{2}{(N-1)(1 - r^2)}$$
With $r$ being the correlational coefficient and $N$ being the number of individuals included.
Finally it was possible to perform the random-effects meta-analysis using the converted dataset as described in the previous steps.
The entire analysis were performed using the \texttt{metafor} package in \texttt{R}.