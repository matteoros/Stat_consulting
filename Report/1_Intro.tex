\newpage
\chapter{Introduction} 

Social anxiety disorder (SAD) is a chronic and difficult-to-cure psychiatric condition that presents as a persistent, disproportional fear of observation and judgment in social settings or social performance situations (\cite{national_collaborating_centre_for_mental_health_uk_social_2013}; \cite{colonnesi_social_2017}); in some contexts, symptoms can extend beyond anxiety around how an individual’s behavior reflects on them-self to include anxiety around how their behavior may reflect on other people (\cite{jefferies_social_2020}). SAD can develop at any time in a person’s life, but it typically emerges during childhood or early adolescence (\cite{national_collaborating_centre_for_mental_health_uk_social_2013}) and is thought to be caused by a combination of genetic and environmental factors (XX).  People experiencing SAD may feel shame, anxiety, and discomfort around interpersonal engagement, fearing humiliation or rejection by their peers (XX).  \\
Categorizing how other traits slot into the disease profile is less straightforward. For example, while psychophysiological symptoms such as heart rate variability (HRV)) or meta-cognitive processes (here: "monitoring of one’s own cognitive processes" \cite{folz_facial_2023}), such as  “worry, rumination, threat-monitoring and self-focused processing” (\cite{nordahl_metacognition_2022}) may occur in response to anxiety-inducing stimuli, anxiety around these psychophysiological or metacognitive responses may itself engender the disorder. That is, while people living with SAD may experience altered heartrate variability or meta-cognitive processes as symptoms of anxiety, anxiety around others noticing and evaluating these responses can also inform an individual’s anticipatory posture towards social interaction. The consequences of SAD can be debilitating and far-reaching, with sufferers experiencing distress in social settings and engaging in avoidance behaviors (XX). In the long term, SAD affects education, professional, and health outcomes (XX). \\
Much research on the role played by psychophysiology and meta-cognition has focused on adult sufferers of SAD, and the extent to which these findings extend to adolescents remains poorly understood. Existing research on therapeutic interventions suggests that early intervention is necessary to prevent the most far-reaching negative outcomes later in life in areas such as education (\cite{vilaplana-perez_much_2021}). These findings are particularly salient given that young people are disproportionately at risk of developing SAD (\cite{jefferies_social_2020}), and as such, recent years have seen an uptick in research on SAD in children and adolescents. However, these small studies have at times produced conflicting results, and the presence of moderators such as age, clinical level of SAD, and measurement tool have made comparison between findings difficult. \\ 
The present research seeks to bring these results together via a meta-analysis of 90 relevant studies with the aim of shedding light on the relationship between meta-cognition, psychophysiology, and SAD. It is our hope that cohesive analysis of study outcomes will support early assessment and intervention for children and adolescents struggling with SAD. To this end, three main Research Questions have been developed:
\begin{enumerate}
    \item Do specific metacognitive processes and psychophysiological patterns contribute to the development of social anxiety disorder in children and adolescents
    \item Does interaction between meta-cognition and psychophysiological phenomena contribute to the development of social anxiety disorder in children and adolescents?
    
    \item To what degree do moderators like age, clinical level, and measurement tool play a role in the development of SAD?
    \item Do the relationships uncovered by the analysis conform to theoretical cognitive-affective model of social anxiety?
   

\end{enumerate}